\documentclass[12pt,a4paper,oneside]{article}
% !TeX spellcheck = en_US
% ---
% PACOTES
% ---

% ---
% Pacotes fundamentais 
% ---
\usepackage{cmap}				% Mapear caracteres especiais no PDF
\usepackage{lmodern}			% Usa a fonte Latin Modern
\usepackage[T1]{fontenc}		% Selecao de codigos de fonte.
\usepackage[utf8]{inputenc}		% Codificacao do documento (conversão automática dos acentos)
\usepackage{indentfirst}		% Indenta o primeiro parágrafo de cada seção.
\usepackage{color}				% Controle das cores
\usepackage{graphicx}			% Inclusão de gráficos
\usepackage{multirow}			% Tabelas
\usepackage{nicefrac}			% Frações
\usepackage{enumerate}
\usepackage{array}
\usepackage{amsmath}
\usepackage{amsfonts}
\usepackage{amsthm}
\usepackage{scalefnt}
\usepackage{url}
\usepackage{longtable,ltcaption} % para as tabelas
\usepackage{pdflscape}
\usepackage{natbib}
\usepackage{hyperref}
% ---
% Pacotes adicionais, usados apenas no âmbito do Modelo Canônico do abnteX2
% ---
\usepackage{lipsum}				% para geração de dummy text
\usepackage[brazil,english]{babel}
% ---
\usepackage{makeidx}
\usepackage[margin=2.5cm]{geometry}
% ---
% Pacotes de citações
% --- Configuração de referências ---
\bibliographystyle{plainnat}
%\hypersetup{hidelinks}
\renewcommand{\baselinestretch}{1.5} % 1.5 denotes double spacing. Changing it will change the spacing


\date{}
\begin{document}
	\title{On the aggregate effects of the payroll tax cut in Brazil: AAA}

\author{
	Caio Matteucci de Andrade Lopes\thanks{Professor: Universidade Federal do Piauí (UFPI). ORCID: \href{https://orcid.org/0000-0002-1754-9062}{0000-0002-1754-9062}. Email: \href{mailto:caiomdealopes@gmail.com}{caiomdealopes@gmail.com}.} \\
	Barbara\thanks{Mestranda: Universidade Federal do Piauí (UFPI). ORCID: \href{https://XXX}{XXX}.Email: \href{mailto:XXX@gmail.com}{XXX@gmail.com}.} \\
	Gabriel\thanks{Mestrando: Universidade Federal do Piauí (UFPI). ORCID: \href{https://XXX}{XXX}. Email: \href{mailto:XXX@gmail.com}{XXX@gmail.com}.} \\
	Leonildo\thanks{Mestrando: Universidade Federal do Piauí (UFPI). ORCID: \href{https://XXX}{XXX}. Email: \href{mailto:XXX@gmail.com}{XXX@gmail.com}.}
}	
	\def \theforeigntitle{}
	
	\maketitle

\selectlanguage{brazil}
\begin{abstract}

		 Este artigo bla bla bla
		 
		\textbf{Classificação JEL}: Ver aqui. \\
		\textbf {Palavras-chave}: e aqui.
\end{abstract}
\selectlanguage{english}
	
	%{
		
		
\begin{abstract}
			This article bla bla in english.
		\\ \\
			 			 
		\textbf{JEL classification:} see here. \\
		\textbf {Keywords}: and here.
\end{abstract}

		
		
		%\cleardoublepage
		% ----------------------------------------------------------
		% Introdução
		% ----------------------------------------------------------
		\section{Introduction}
	
		\textbf{citar com citep(final de paragrafo) e citet (meio de paragrafo)}
		% ----------------------------------------------------------
		% Capitulo de textual  
		% ----------------------------------------------------------
		\section{Related Work}
		
		
				
				% ---------------------------------------------------------------------------------
		\section{Methodology}
						
		\subsection{Econometric Model}
		
				
		\subsection{Empirical Approach}
		
		
		
		\section{Data}
		
		\section{Results \label{rp}}
		
		
		\subsection{Main Results and Discussion}
		
				
		\subsection{Robustness Checks}
						
		
		% ----------------------------------------------------
		% Conclusão
		\section{Concluding Remarks}
				
		
				
		%	\clearpage
		
		% ----------------------------------------------------------
		% Referências bibliográficas
		% ----------------------------------------------------------
		\bibliography{mybibfile.bib}
		
		\cleardoublepage
		
		\appendix
		%\section{Appendix}
		
 		
	\end{document}
